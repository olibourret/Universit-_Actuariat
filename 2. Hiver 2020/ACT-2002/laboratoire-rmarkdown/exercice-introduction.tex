\PassOptionsToPackage{unicode=true}{hyperref} % options for packages loaded elsewhere
\PassOptionsToPackage{hyphens}{url}
%
\documentclass[]{article}
\usepackage{lmodern}
\usepackage{amssymb,amsmath}
\usepackage{ifxetex,ifluatex}
\usepackage{fixltx2e} % provides \textsubscript
\ifnum 0\ifxetex 1\fi\ifluatex 1\fi=0 % if pdftex
  \usepackage[T1]{fontenc}
  \usepackage[utf8]{inputenc}
  \usepackage{textcomp} % provides euro and other symbols
\else % if luatex or xelatex
  \usepackage{unicode-math}
  \defaultfontfeatures{Ligatures=TeX,Scale=MatchLowercase}
\fi
% use upquote if available, for straight quotes in verbatim environments
\IfFileExists{upquote.sty}{\usepackage{upquote}}{}
% use microtype if available
\IfFileExists{microtype.sty}{%
\usepackage[]{microtype}
\UseMicrotypeSet[protrusion]{basicmath} % disable protrusion for tt fonts
}{}
\IfFileExists{parskip.sty}{%
\usepackage{parskip}
}{% else
\setlength{\parindent}{0pt}
\setlength{\parskip}{6pt plus 2pt minus 1pt}
}
\usepackage{hyperref}
\hypersetup{
            pdftitle={Traduction du manuel de statistiques},
            pdfauthor={Thomas Paré-Bourque},
            pdfborder={0 0 0},
            breaklinks=true}
\urlstyle{same}  % don't use monospace font for urls
\usepackage[margin=1in]{geometry}
\usepackage{graphicx,grffile}
\makeatletter
\def\maxwidth{\ifdim\Gin@nat@width>\linewidth\linewidth\else\Gin@nat@width\fi}
\def\maxheight{\ifdim\Gin@nat@height>\textheight\textheight\else\Gin@nat@height\fi}
\makeatother
% Scale images if necessary, so that they will not overflow the page
% margins by default, and it is still possible to overwrite the defaults
% using explicit options in \includegraphics[width, height, ...]{}
\setkeys{Gin}{width=\maxwidth,height=\maxheight,keepaspectratio}
\setlength{\emergencystretch}{3em}  % prevent overfull lines
\providecommand{\tightlist}{%
  \setlength{\itemsep}{0pt}\setlength{\parskip}{0pt}}
\setcounter{secnumdepth}{0}
% Redefines (sub)paragraphs to behave more like sections
\ifx\paragraph\undefined\else
\let\oldparagraph\paragraph
\renewcommand{\paragraph}[1]{\oldparagraph{#1}\mbox{}}
\fi
\ifx\subparagraph\undefined\else
\let\oldsubparagraph\subparagraph
\renewcommand{\subparagraph}[1]{\oldsubparagraph{#1}\mbox{}}
\fi

% set default figure placement to htbp
\makeatletter
\def\fps@figure{htbp}
\makeatother


\title{Traduction du manuel de statistiques}
\author{Thomas Paré-Bourque}
\date{16/04/2020}

\begin{document}
\maketitle

\#Section 8.5 : Intervalle de confiance Un estimateur d'intervalle est
une règle spécifiant la méthode d'utilisation des mesures de
l'échantillon pour calculer deux nombres qui forment les extrémités de
l'intervalle. Idéalement, le l'intervalle résultant aura deux
propriétés: Premièrement, il contiendra le paramètre cible \(\theta\);
deuxièmement, il sera relativement étroit. Un ou les deux points finaux
de l'intervalle, étant Les fonctions des mesures de l'échantillon
varient de façon aléatoire d'un échantillon à l'autre. Ainsi, la
longueur et l'emplacement de l'intervalle sont des quantités aléatoires,
et nous ne pouvons pas nous assurer que le paramètre cible (fixe)
\(\theta\) se situera entre les extrémités de tout intervalle unique
calculé à partir d'un seul échantillon. Cela étant, notre objectif est
pour trouver un estimateur d'intervalle capable de générer des
intervalles étroits qui ont une forte probabilité d'enfermer \(\theta\).
Les estimateurs d'intervalle sont communément appelés intervalles de
confiance. Le haut et le bas les points limites d'un intervalle de
confiance sont appelés les limites de confiance supérieure et
inférieure, respectivement. La probabilité qu'un intervalle de confiance
(aléatoire) englobe \(\theta\) (une quantité fixe) est appelé
coefficient de confiance. D'un point de vue pratique, le coefficient de
confiance identifie la fraction du temps, en échantillonnage répété, que
les intervalles construits contiendront le paramètre cible \(\theta\).
Si nous savons que le coefficient de confiance associé à notre
estimateur est élevé, nous pouvons être très confiant que tout
intervalle de confiance, construit en utilisant les résultats d'un seul
échantillon, contiendra \(\theta\). Supposons que \(\widehat{\theta}_L\)
et \(\widehat{\theta}_U\) sont les limites de confiance inférieures et
supérieures (aléatoires), respectivement, pour un paramètre \(\theta\).
Puis si

ÉCRITURE MATH

la probabilité \((1-\alpha)\) est le coefficient de confiance.
L'intervalle aléatoire résultant, défini par {[}\(\widehat{\theta}_L\),
\(\widehat{\theta}_U\){]} , est appelé un intervalle de confiance
bilatéral. Il est également possible de former un intervalle de
confiance unilatéral tel que

ÉCRITURE MATH

Bien que seul {[}\(\widehat{\theta}_L\) soit aléatoire dans ce cas,
l'intervalle de confiance est {[}\(\widehat{\theta}_L\), \(\infty\)). De
même, nous pourrions avoir un intervalle de confiance unilatéral
supérieur tel que ÉCRITURE MATH L'intervalle de confiance implicite ici
est \$(−\infty, \widehat{\theta}\_U{]}\$. Une méthode très utile pour
trouver des intervalles de confiance est appelée la méthode du pivot.
Cette méthode dépend de la recherche d'une quantité pivot qui possède
deux caractéristiques:\\
1. Elle est fonction des mesures de l'échantillon et du paramètre
inconnu \(\theta\), où \(\theta\) est la seule quantité inconnue.\\
2. Sa distribution de probabilité ne dépend pas du paramètre
\(\theta\).\\
Si la distribution de probabilité de la quantité pivot est connue, la
logique suivante peut être utilisé pour former l'estimation d'intervalle
souhaitée. Si \(Y\) est une variable aléatoire, \(c \ge 0\) (pas égal,
juste pas capable de trouver ce symbole) est une constante, et
\(P(a \leq Y \leq b) = 0.7\); alors certainement
\(P(ca \leq cY \leq cb) = .7\). De même, pour toute constante \(d\),
\(P(a+d \leq Y+d \leq b+d) = 0.7\). Autrement dit, la probabilité de
l'événement \((a \leq Y \leq b)\) n'est pas affecté par un changement
d'échelle ou une translation de Y. Ainsi, si nous connaissons la
distribution de probabilité d'une quantité pivot, nous pourrons
peut-être utiliser des opérations comme celles-ci pour former
l'estimateur d'intervalle souhaité. Nous illustrons cette méthode dans
les exemples suivants.

EXEMPLES ÉCRITURE MATH

Les deux exemples précédents illustrent l'utilisation de la méthode
pivot pour trouver limites de confiance pour les paramètres inconnus.
Dans chaque cas, les estimations d'intervalle ont été développés sur la
base d'une seule observation de la distribution. Ces exemples ont été
introduites principalement pour illustrer la méthode pivot. Dans le
reste sections de ce chapitre, nous utilisons cette méthode en
conjonction avec les distributions d'échantillonnage présenté au
chapitre 7 pour développer des estimations d'intervalle de plus grande
pratique importance. \#Section 8.6 : Intervalle de confiance pour grands
échantillons

Dans la section 8.3, nous avons présenté quelques estimateurs ponctuels
sans biais pour les paramètres \(\mu\), p, \(\mu_1 - \mu_2\) et
\(p_1 - p_2\). Comme nous l'avons indiqué dans cette section, pour de
grands échantillons, tous ces points les estimateurs ont des
distributions d'échantillonnage approximativement normales avec des
erreurs donnée dans le tableau 8.1. Autrement dit, dans les conditions
de la section 8.3, si le paramètre cible \(\theta\) est \(\mu\), \(p\),
\(\mu_1 - \mu_2\), ou \(p_1 - p_2\), alors pour les grands échantillons,
\(Z = \frac{\widehat{\theta}- \theta}{\sigma_\widehat{\theta}}\) possède
approximativement une distribution normale standard. Par conséquent,
\(Z = \frac{\widehat{\theta}- \theta}{\sigma_\widehat{\theta}}\) forme
(au moins approximativement) une quantité pivot, et la méthode pivot
peut être utilisé pour développer des intervalles de confiance pour le
paramètre cible \(\theta\).

EXEMPLE 8.6

Par des arguments analogues, nous pouvons déterminer que
\(100*(1 − \alpha)\%\) de confiance unilatérale les limites, souvent
appelées respectivement limites supérieure et inférieure, sont données
par \(100*(1 − \alpha)\%\) borne inférieure pour
\(\theta = \widehat{\theta} - z*\alpha*\sigma_\widehat{\theta}\)
\(100*(1 − \alpha)\%\) borne supérieure pour
\(\theta = \widehat{\theta} + z*\alpha*\sigma_\widehat{\theta}\)
Supposons que nous calculons à la fois une borne inférieure de
\(100*(1 − \alpha)\%\) et une borne supérieure de \(100*(1 − \alpha)\%\)
borne pour \(\theta\). Nous décidons ensuite d'utiliser ces deux bornes
pour former un intervalle de confiance pour \(\theta\). Quel sera le
coefficient de confiance de cet intervalle? Un coup d'oeil sur le
précédent confirme que la combinaison des limites inférieures et
supérieures, chacune avec confiance le coefficient \((1-\alpha)\) ,
donne un intervalle bilatéral avec le coefficient de confiance
\((1-2\alpha)\). Dans les conditions décrites à la section 8.3, les
résultats donnés plus haut dans la section peuvent être utilisés pour
trouver des intervalles de confiance à large échantillon (unilatéral ou
bilatéral) pour \(\mu\), p, \(\mu_1 - \mu_2\) et \(p_1 - p_2\). Les
exemples suivants illustrent des applications de la méthode générale
développée dans l'exemple 8.6.

EXEMPLE 8.7 et EXEMPLE 8.8 Notez que cet intervalle de confiance
contient zéro. Ainsi, une valeur nulle pour la différence en proportions
\((p_1 - p_2)\) est ``crédible'' (à un niveau de confiance d'environ
98\%) sur la base des données observées. Cependant, l'intervalle inclut
également la valeur .1. Ainsi, .1 représente une autre valeur de
\((p_1 - p_2)\) qui est ``crédible'' sur la base de la données que nous
avons analysées. Nous terminons cette section par une étude empirique de
la performance du procédure d'estimation d'intervalle sur un grand
échantillon pour une seule proportion de population p, basée sur Y, le
nombre de succès observés lors de n essais dans une expérience
binomiale. Dans ce cas, \(\theta = p\);
\(\widehat{\theta} = \widehat{p} = \frac{Y}{n}\) et
\(\sigma_\widehat{\theta} = \sigma_\widehat{p} = \sqrt{\frac{p(1-p)}{n}} \approx \sqrt{\frac{\widehat{p}(1-\widehat{p})}{n}}\)
(Comme dans la section 8.3,
\(\sqrt{\frac{\widehat{p}(1-\widehat{p})}{n}}\) fournit une bonne
approximation de \(\sigma_\widehat{p}\).) Les limites de confiance sont
alors \[
\widehat{\theta}_L = \widehat{p}-z_{\alpha/2}\sqrt{\frac{\widehat{p}(1-\widehat{p})}{n}}
\space et \space
\widehat{\theta}_U = \widehat{p}+z_{\alpha/2}\sqrt{\frac{\widehat{p}(1-\widehat{p})}{n}}.
\] La figure 8.8 montre les résultats de 24 expériences binomiales
indépendantes, chacune basée sur 35 essais lorsque la vraie valeur de
\(p = 0,5\). Pour chacune des expériences, nous avons calculé le nombre
de succès y, la valeur de \(\widehat{p} = \frac{y}{35}\) 5 et la
confiance correspondante à l'intervalle est de \(95\%\), en utilisant la
formule \(\widehat{p} \pm 1,96\) (Notez que \(z_{0.25} = 1,96\).) Dans
la première expérience binomiale, nous avons observé \(y = 18\),
\(\widehat{p}= 18/35 = 0,5143\), et
\(\sigma_\widehat{p} = \sqrt{\frac{\widehat{p}(1-\widehat{p})}{n}} = \sqrt{\frac{(0.5143)(0.4857)}{35}} = 0.0845\).
Ainsi, l'intervalle obtenu dans la première expérience est de
\(0.5143 \pm 1.96(0.0845)\) ou \((0.34487, 0.6799)\). L'estimation pour
\(p\) de la première expérience est indiqué par le grand point le plus
bas sur la figure 8.8, etl'intervalle de confiance résultant est donné
par la ligne horizontale passant par ce point. La ligne verticale
indique la vraie valeur de \(p\), \(0.5\) dans ce cas. Notez que
l'intervalle obtenu dans le premier essai (de taille 35) contient en
fait la vraie valeur de la population proportion \(p\). Les 23
intervalles de confiance restants contenus dans cette petite simulation
sont donnés par le reste des lignes horizontales de la figure 8.8. Notez
que chaque intervalle individuel soit contient la vraie valeur de \(p\),
soit il n'en contient pas. Cependant, la vraie valeur de \(p\) est
contenue dans 23 des 24 (95,8\%) intervalles observés. Si la même
procédure était utilisée plusieurs fois, chaque intervalle individuel
serait contiennent ou ne contiennent pas la vraie valeur de p, mais le
pourcentage de tous les intervalles qui la capture p serait très proche
de 95\%. Vous êtes «à 95\% confiant» que l'intervalle contient le
paramètre car l'intervalle a été obtenu à l'aide d'une procédure qui
génère des intervalles qui contiennent le paramètre environ 95\% des
fois où est utilisée. L'applet ConfidenceIntervalP (accessible à
academic.cengage.com/statistics/ wackerly) a été utilisé pour produire
la figure 8.8. Que se passe-t-il si différentes valeurs de \(n\) ou
différents coefficients de confiance sont-ils utilisés? Obtenons-nous
des résultats similaires si la vraie valeur de \(p\) est autre chose que
\(0.5\)? Plusieurs des exercices suivants permettront vous utilisez
l'applet pour répondre à des questions comme celles-ci. Dans cette
section, nous avons utilisé la méthode pivot pour dériver la confiance
des grands échantillons intervalles pour les paramètres
\(\mu, p,\mu_1-\mu_2, p_1-p_2\) dans les conditions de Section 8.3. La
formule clé est
\(\widehat{\theta} \pm z_{\alpha/2}\sigma_{\widehat{\theta}}\) où les
valeurs de \(\widehat{\theta}\) et \(\sigma_{\widehat{\theta}}\) sont
celles indiquées dans le tableau 8.1. Lorsque \(\theta = \mu\) est la
cible , alors \(\widehat{\theta}= \bar{Y}\) et
\(\sigma^2_{\widehat{\theta}} = \sigma^2/n\), où \(\sigma^2\) est la
variance de la population. Si la la vraie valeur de \(\sigma^2\) est
connue, cette valeur doit être utilisée pour calculer la confiance
intervalle. Si \(\sigma^2\) n'est pas connu et \(n\) est grand, il n'y a
pas de perte de précision sérieuse si \(s^2\) remplace \(\sigma^2\) dans
la formule de l'intervalle de confiance. De même, si \(\sigma^2_1\) et
\(\sigma^2_2\) sont inconnus et \(n_1\) et \(n_2\) sont grands,
\(s^2_1\) et \(s^2_2\) peuvent être substitués à ceux-ci valeurs de la
formule pour un intervalle de confiance à grand échantillon pour
\(\theta = \mu_1 - \mu_2\). Lorsque \(\theta =p\) est le paramètre
cible, alors \(\widehat{\theta} = \widehat{p}\) et
\(\sigma_{\widehat{p}} = \sqrt{pq/n}\). Parce que \(p\) est le paramètre
cible inconnu, \(\sigma_{\widehat{p}}\) ne peut pas être évalué. Si
\(n\) est grand et que nous substituons \(\widehat{p}\) pour \(p\) (et
\(\widehat{q} = 1− \widehat{p}\) pour \(q\)) dans la formule pour
\(\sigma_{\widehat{p}}\), cependant, la confiance qui en résulte,
l'intervalle aura approximativement le coefficient de confiance indiqué.
Pour des grands \(n_1\) et \(n_2\), des énoncés similaires sont valables
lorsque \(\widehat{p}_1\) et \(\widehat{p}_2\) sont utilisés pour
estimer \(p_1\) et \(p_2\), respectivement, dans la formule pour
\(\sigma^2_{\widehat{p}_1 - \widehat{p}_2}\). La justification théorique
de ces substitutions sera prévu à la section 9.3.

\end{document}
