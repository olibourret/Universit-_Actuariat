\PassOptionsToPackage{unicode=true}{hyperref} % options for packages loaded elsewhere
\PassOptionsToPackage{hyphens}{url}
%
\documentclass[]{article}
\usepackage{lmodern}
\usepackage{amssymb,amsmath}
\usepackage{ifxetex,ifluatex}
\usepackage{fixltx2e} % provides \textsubscript
\ifnum 0\ifxetex 1\fi\ifluatex 1\fi=0 % if pdftex
  \usepackage[T1]{fontenc}
  \usepackage[utf8]{inputenc}
  \usepackage{textcomp} % provides euro and other symbols
\else % if luatex or xelatex
  \usepackage{unicode-math}
  \defaultfontfeatures{Ligatures=TeX,Scale=MatchLowercase}
\fi
% use upquote if available, for straight quotes in verbatim environments
\IfFileExists{upquote.sty}{\usepackage{upquote}}{}
% use microtype if available
\IfFileExists{microtype.sty}{%
\usepackage[]{microtype}
\UseMicrotypeSet[protrusion]{basicmath} % disable protrusion for tt fonts
}{}
\IfFileExists{parskip.sty}{%
\usepackage{parskip}
}{% else
\setlength{\parindent}{0pt}
\setlength{\parskip}{6pt plus 2pt minus 1pt}
}
\usepackage{hyperref}
\hypersetup{
            pdftitle={Documentation - Fonction irr en R},
            pdfauthor={William Perron, Simon Veilleux, Félix Laflamme, Olivier Bourret},
            pdfborder={0 0 0},
            breaklinks=true}
\urlstyle{same}  % don't use monospace font for urls
\usepackage[margin=1in]{geometry}
\usepackage{color}
\usepackage{fancyvrb}
\newcommand{\VerbBar}{|}
\newcommand{\VERB}{\Verb[commandchars=\\\{\}]}
\DefineVerbatimEnvironment{Highlighting}{Verbatim}{commandchars=\\\{\}}
% Add ',fontsize=\small' for more characters per line
\usepackage{framed}
\definecolor{shadecolor}{RGB}{248,248,248}
\newenvironment{Shaded}{\begin{snugshade}}{\end{snugshade}}
\newcommand{\AlertTok}[1]{\textcolor[rgb]{0.94,0.16,0.16}{#1}}
\newcommand{\AnnotationTok}[1]{\textcolor[rgb]{0.56,0.35,0.01}{\textbf{\textit{#1}}}}
\newcommand{\AttributeTok}[1]{\textcolor[rgb]{0.77,0.63,0.00}{#1}}
\newcommand{\BaseNTok}[1]{\textcolor[rgb]{0.00,0.00,0.81}{#1}}
\newcommand{\BuiltInTok}[1]{#1}
\newcommand{\CharTok}[1]{\textcolor[rgb]{0.31,0.60,0.02}{#1}}
\newcommand{\CommentTok}[1]{\textcolor[rgb]{0.56,0.35,0.01}{\textit{#1}}}
\newcommand{\CommentVarTok}[1]{\textcolor[rgb]{0.56,0.35,0.01}{\textbf{\textit{#1}}}}
\newcommand{\ConstantTok}[1]{\textcolor[rgb]{0.00,0.00,0.00}{#1}}
\newcommand{\ControlFlowTok}[1]{\textcolor[rgb]{0.13,0.29,0.53}{\textbf{#1}}}
\newcommand{\DataTypeTok}[1]{\textcolor[rgb]{0.13,0.29,0.53}{#1}}
\newcommand{\DecValTok}[1]{\textcolor[rgb]{0.00,0.00,0.81}{#1}}
\newcommand{\DocumentationTok}[1]{\textcolor[rgb]{0.56,0.35,0.01}{\textbf{\textit{#1}}}}
\newcommand{\ErrorTok}[1]{\textcolor[rgb]{0.64,0.00,0.00}{\textbf{#1}}}
\newcommand{\ExtensionTok}[1]{#1}
\newcommand{\FloatTok}[1]{\textcolor[rgb]{0.00,0.00,0.81}{#1}}
\newcommand{\FunctionTok}[1]{\textcolor[rgb]{0.00,0.00,0.00}{#1}}
\newcommand{\ImportTok}[1]{#1}
\newcommand{\InformationTok}[1]{\textcolor[rgb]{0.56,0.35,0.01}{\textbf{\textit{#1}}}}
\newcommand{\KeywordTok}[1]{\textcolor[rgb]{0.13,0.29,0.53}{\textbf{#1}}}
\newcommand{\NormalTok}[1]{#1}
\newcommand{\OperatorTok}[1]{\textcolor[rgb]{0.81,0.36,0.00}{\textbf{#1}}}
\newcommand{\OtherTok}[1]{\textcolor[rgb]{0.56,0.35,0.01}{#1}}
\newcommand{\PreprocessorTok}[1]{\textcolor[rgb]{0.56,0.35,0.01}{\textit{#1}}}
\newcommand{\RegionMarkerTok}[1]{#1}
\newcommand{\SpecialCharTok}[1]{\textcolor[rgb]{0.00,0.00,0.00}{#1}}
\newcommand{\SpecialStringTok}[1]{\textcolor[rgb]{0.31,0.60,0.02}{#1}}
\newcommand{\StringTok}[1]{\textcolor[rgb]{0.31,0.60,0.02}{#1}}
\newcommand{\VariableTok}[1]{\textcolor[rgb]{0.00,0.00,0.00}{#1}}
\newcommand{\VerbatimStringTok}[1]{\textcolor[rgb]{0.31,0.60,0.02}{#1}}
\newcommand{\WarningTok}[1]{\textcolor[rgb]{0.56,0.35,0.01}{\textbf{\textit{#1}}}}
\usepackage{graphicx,grffile}
\makeatletter
\def\maxwidth{\ifdim\Gin@nat@width>\linewidth\linewidth\else\Gin@nat@width\fi}
\def\maxheight{\ifdim\Gin@nat@height>\textheight\textheight\else\Gin@nat@height\fi}
\makeatother
% Scale images if necessary, so that they will not overflow the page
% margins by default, and it is still possible to overwrite the defaults
% using explicit options in \includegraphics[width, height, ...]{}
\setkeys{Gin}{width=\maxwidth,height=\maxheight,keepaspectratio}
\setlength{\emergencystretch}{3em}  % prevent overfull lines
\providecommand{\tightlist}{%
  \setlength{\itemsep}{0pt}\setlength{\parskip}{0pt}}
\setcounter{secnumdepth}{0}
% Redefines (sub)paragraphs to behave more like sections
\ifx\paragraph\undefined\else
\let\oldparagraph\paragraph
\renewcommand{\paragraph}[1]{\oldparagraph{#1}\mbox{}}
\fi
\ifx\subparagraph\undefined\else
\let\oldsubparagraph\subparagraph
\renewcommand{\subparagraph}[1]{\oldsubparagraph{#1}\mbox{}}
\fi

% set default figure placement to htbp
\makeatletter
\def\fps@figure{htbp}
\makeatother


\title{Documentation - Fonction irr en R}
\author{\emph{William Perron}, \emph{Simon Veilleux}, \emph{Félix Laflamme},
\emph{Olivier Bourret}}
\date{\emph{28/02/2020}}

\begin{document}
\maketitle

\begin{Shaded}
\begin{Highlighting}[]
\NormalTok{irr <-}\StringTok{ }\ControlFlowTok{function}\NormalTok{(x)}
\NormalTok{\{}
    \ControlFlowTok{if}\NormalTok{(}\KeywordTok{all}\NormalTok{(x }\OperatorTok{>=}\StringTok{ }\DecValTok{0}\NormalTok{) }\OperatorTok{|}\StringTok{ }\KeywordTok{all}\NormalTok{(x }\OperatorTok{<=}\DecValTok{0}\NormalTok{))}
    \KeywordTok{stop}\NormalTok{(}\StringTok{'Tous les flux financiers sont du même signe'}\NormalTok{)}
    
    \ControlFlowTok{if}\NormalTok{(}\KeywordTok{sum}\NormalTok{(}\KeywordTok{abs}\NormalTok{(}\KeywordTok{Im}\NormalTok{(}\KeywordTok{polyroot}\NormalTok{(x))) }\OperatorTok{<}\StringTok{ }\NormalTok{.Machine}\OperatorTok{$}\NormalTok{double.eps}\OperatorTok{^}\FloatTok{0.5}\NormalTok{) }\OperatorTok{>}\StringTok{ }\DecValTok{1}\NormalTok{)}
        \KeywordTok{warning}\NormalTok{(}\StringTok{"Plus d'un changement de signe dans les flux nets, le taux de rendement peut ne pas être unique"}\NormalTok{)}
   
\NormalTok{    i <-}\StringTok{ }\NormalTok{(}\KeywordTok{Re}\NormalTok{(}\KeywordTok{polyroot}\NormalTok{(x))[}\KeywordTok{abs}\NormalTok{(}\KeywordTok{Im}\NormalTok{(}\KeywordTok{polyroot}\NormalTok{(x))) }\OperatorTok{<}\StringTok{ }\NormalTok{.Machine}\OperatorTok{$}\NormalTok{double.eps}\OperatorTok{^}\FloatTok{0.5}\NormalTok{])}\OperatorTok{^}\NormalTok{(}\OperatorTok{-}\DecValTok{1}\NormalTok{) }\OperatorTok{-}\StringTok{ }\DecValTok{1}
    
\NormalTok{   i[i }\OperatorTok{>}\StringTok{ }\DecValTok{-1}\NormalTok{]}
\NormalTok{\}}
\end{Highlighting}
\end{Shaded}

\begin{verbatim}
## Warning in irr(flux): Plus d'un changement de signe dans les flux nets, le taux
## de rendement peut ne pas être unique
\end{verbatim}

Signature

\begin{verbatim}
irr(flux)
\end{verbatim}

Description La fonction \(irr\) calcule le taux (ou un des taux
possibles) d'intérêt satisfaisant une équation de valeur de flux
financiers à des temps différents.

Arguments \(flux\) : vecteur de tous les flux financiers des temps \(0\)
à \(n\).

Valeur nombre réel supérieur à \(-1\)

Format des flux financiers

Considérons l'exemple précédent des flux financiers tel que

\begin{verbatim}
-7+4v^3=3v+2v^2+3v^5+v^6, où v=(1+i)^(-1)
\end{verbatim}

En évaluant la fonction \(irr\) avec lesdits flux financiers, le
résultat affiché est -0.0909397.

\hypertarget{correction}{%
\subsubsection{Correction}\label{correction}}

\hypertarget{commentaires-avertissement-pour-le-changement-de-signes-mal-duxe9fini-espaces-pas-constants-documentation-difficile-uxe0-lire-document-produit-est-mal-organisuxe9}{%
\subsection{Commentaires: Avertissement pour le changement de signes mal
défini, espaces pas constants, documentation difficile à lire, document
produit est mal
organisé}\label{commentaires-avertissement-pour-le-changement-de-signes-mal-duxe9fini-espaces-pas-constants-documentation-difficile-uxe0-lire-document-produit-est-mal-organisuxe9}}

\hypertarget{r-810}{%
\section{R : 8/10}\label{r-810}}

\hypertarget{a-55}{%
\section{A : 5/5}\label{a-55}}

\hypertarget{s-45}{%
\section{S : 4/5}\label{s-45}}

\hypertarget{d-45}{%
\section{D : 4/5}\label{d-45}}

\hypertarget{c-35}{%
\section{C : 3/5}\label{c-35}}

\hypertarget{solution}{%
\subsection{Solution :}\label{solution}}

\hypertarget{je-ne-fournis-que-la-solution-pour-la-fonction-irr.}{%
\section{Je ne fournis que la solution pour la fonction
irr.}\label{je-ne-fournis-que-la-solution-pour-la-fonction-irr.}}

\hypertarget{la-solution-ci-dessous-est-donnuxe9e-uxe0-titre-indicatif.-il-peut-exister-dautres}{%
\section{La solution ci-dessous est donnée à titre indicatif. Il peut
exister
d'autres}\label{la-solution-ci-dessous-est-donnuxe9e-uxe0-titre-indicatif.-il-peut-exister-dautres}}

\hypertarget{solutions-tout-aussi-valables.}{%
\section{solutions tout aussi
valables.}\label{solutions-tout-aussi-valables.}}

\hypertarget{section-1}{%
\section{}\label{section-1}}

\hypertarget{pour-le-test-sur-les-changements-de-signe-on-supprime-les-0-du-vecteur}{%
\section{Pour le test sur les changements de signe, on supprime les 0 du
vecteur}\label{pour-le-test-sur-les-changements-de-signe-on-supprime-les-0-du-vecteur}}

\hypertarget{des-flux-pour-contourner-les-cas-ouxf9-la-suxe9rie-passerait-par-0-avant-de-changer}{%
\section{des flux pour contourner les cas où la série passerait par 0
avant de
changer}\label{des-flux-pour-contourner-les-cas-ouxf9-la-suxe9rie-passerait-par-0-avant-de-changer}}

\hypertarget{de-signe-par-exemple-5-0-3.-un-tel-cas-est-difficile-uxe0-distinguer-dun}{%
\section{de signe (par exemple : −5, 0, 3). Un tel cas est difficile à
distinguer
d'un}\label{de-signe-par-exemple-5-0-3.-un-tel-cas-est-difficile-uxe0-distinguer-dun}}

\hypertarget{passage-par-0-sans-changement-de-signe-par-exemple-5-0-2.}{%
\section{passage par 0 sans changement de signe (par exemple : −5, 0,
−2).}\label{passage-par-0-sans-changement-de-signe-par-exemple-5-0-2.}}

irr \textless{}- function(x) \{ if (all(x \textgreater{}= 0)
\textbar{}\textbar{} all(x \textless{}= 0)) stop(``tous les flux
financiers sont du même signe'') if (sum(diff(sign(x{[}x != 0{]})) != 0)
\textgreater{} 1) warning(``plus d'un changement de signe dans les flux
nets\n le taux de rendement peut ne pas être unique'') r \textless{}-
polyroot(x) i \textless{}- 1/Re(r){[}abs(Im(r)) \textless{}
.Machine\$double.eps\^{}0.5{]} - 1 i{[}i \textgreater{} -1{]} \}

\end{document}
