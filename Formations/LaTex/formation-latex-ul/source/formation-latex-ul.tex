%%% Copyright (C) 2020 Vincent Goulet
%%%
%%% Ce fichier fait partie du projet
%%% «Rédaction avec LaTeX»
%%% https://gitlab.com/vigou3/formation-latex-ul
%%%
%%% Cette création est mise à disposition sous licence
%%% Attribution-Partage dans les mêmes conditions 4.0
%%% International de Creative Commons.
%%% https://creativecommons.org/licenses/by-sa/4.0/

\documentclass[letterpaper,11pt,x11names,english,french]{memoir}
  \usepackage{natbib,url,bibentry}
  \usepackage{babel}
  \usepackage[autolanguage]{numprint}
  \usepackage{amsmath,amsthm}
  \usepackage[shortlabels]{enumitem}
  \usepackage{graphicx}
  \usepackage{manfnt}                  % \mantriangleright (puce)
  \usepackage{dirtree}                 % arbre pour exercice sur \include
  \usepackage{metalogo}                % logo de \XeLaTeX
  \usepackage{mflogo}                  % logo de Metafont
  \usepackage{relsize}                 % \smaller et al.
  \usepackage{fontawesome5}            % plusieurs icônes
  \usepackage{awesomebox}              % boites info, important, etc.
  \usepackage{applekeys}               % touches Mac
  \usepackage{answers}                 % exercices et solutions
  \usepackage{listings}                % code informatique
  \usepackage[absolute]{textpos}       % disposition d'images
  \usepackage{ifthen}                  % exécution conditionnelle
  \usepackage{soul}                    % base pour boites autour des caractères


  %%% =============================
  %%%  Informations de publication
  %%% =============================
  \title{Rédaction avec \LaTeX}
  \author{Vincent Goulet}
  \renewcommand{\year}{2020}
  \renewcommand{\month}{10}
  \newcommand{\ctanurl}{https://ctan.org/pkg/formation-latex-ul/}
  \newcommand{\reposurl}{https://gitlab.com/vigou3/formation-latex-ul/}

  %%% ===================
  %%%  Style du document
  %%% ===================

  %% Polices de caractères
  \usepackage{fontspec}
  \usepackage{unicode-math}
  \defaultfontfeatures
  {
    Scale = 0.92
  }
  \setmainfont{Lucida Bright OT}
  [
    Ligatures = TeX,
    Numbers = OldStyle
  ]
  \setmathfont{Lucida Bright Math OT}
  \setmonofont{Lucida Grande Mono DK}
  \setsansfont{FiraSans}
  [
    Extension = .otf,
    UprightFont = *-Book,
    ItalicFont = *-BookItalic,
    BoldFont = *-SemiBold,
    BoldItalicFont = *-SemiBoldItalic,
    Scale = 0.95,
    Numbers = OldStyle
  ]
  \newfontfamily\fullcaps{FiraSans}
  [
    Extension = .otf,
    UprightFont = *-Book,
    Scale = 0.95,
    Numbers = Uppercase
  ]
  \usepackage[babel=true]{microtype}
  \usepackage{icomma}

  %% Polices additionnelles pour le chapitre trucs et astuces
  \newfontfamily\CM[Ligatures=TeX]{cmunrm.otf}        % Computer Modern
  \newfontfamily\Times{texgyretermes-regular.otf}     % Times
  \newfontfamily\Palatino{texgyrepagella-regular.otf} % Palatino
  \newfontfamily\Bookman{texgyrebonum-regular.otf}    % Bookman
  \newfontfamily\NewCent{texgyreschola-regular.otf}   % New Cent. Sch.
  \newfontfamily\Charter{XCharter-Roman.otf}          % Charter
  \newfontfamily\Helvet{texgyreheros-regular.otf}     % Helvetica

  %% Couleurs
  \usepackage{xcolor}
  \definecolor{comments}{rgb}{0.5,0.55,0.6} % commentaires
  \definecolor{link}{rgb}{0,0.4,0.6}        % liens internes
  \definecolor{url}{rgb}{0.6,0,0}           % liens externes
  \definecolor{citation}{rgb}{0,0.5,0}      % citations
  \definecolor{ULlinkcolor}{rgb}{0,0,0.3}   % de ulthese.cls
  \definecolor{rouge}{rgb}{0.90,0,0.1}      % rouge bandeau UL
  \definecolor{or}{rgb}{1,0.8,0}            % or bandeau UL

  %% Hyperliens
  \usepackage{hyperref}
  \hypersetup{%
    pdfauthor = \theauthor,
    pdftitle = {Rédaction avec LaTeX},
    colorlinks = {true},
    linktocpage = {true},
    urlcolor = {url},
    linkcolor = {link},
    citecolor = {citation},
    pdfpagemode = {UseOutlines},
    pdfstartview = {Fit}}
  \setlength{\XeTeXLinkMargin}{1pt}

  %% Affichage de la table des matières du PDF
  \usepackage{bookmark}
  \bookmarksetup{%
    open = true,
    depth = 3,
    numbered = true}

  %% Étiquettes de \autoref (redéfinitions compatibles avec babel).
  %% Attention! Les % à la fin des lignes sont importants sinon des
  %% blancs apparaissent dès que la commande \selectlanguage est
  %% utilisée, notamment dans la bibliographie.
  \addto\extrasfrench{%
    \def\subsectionautorefname{section}%
    \def\figureautorefname{figure}%
    \def\tableautorefname{tableau}%
    \def\exempleautorefname{exemple}%
    \def\exerciceautorefname{exercice}%
    \def\appendixautorefname{annexe}%
  }

  %% Table des matières et al. (inspiré de classicthesis.sty)
  \renewcommand{\cftchapterleader}{\hspace{1.5em}}
  \renewcommand{\cftchapterafterpnum}{\cftparfillskip}
  \renewcommand{\cftsectionleader}{\hspace{1.5em}}
  \renewcommand{\cftsectionafterpnum}{\cftparfillskip}
  \renewcommand{\cfttableleader}{\hspace{1.5em}}
  \renewcommand{\cfttableafterpnum}{\cftparfillskip}
  \renewcommand{\cftfigureleader}{\hspace{1.5em}}
  \renewcommand{\cftfigureafterpnum}{\cftparfillskip}
  \setlength{\cftfigurenumwidth}{3.2em}

  %% Titres des chapitres
  \chapterstyle{hangnum}
  \renewcommand{\chaptitlefont}{\normalfont\Huge\sffamily\bfseries\raggedright}

  %% Marges, entêtes et pieds de page
  \setlength{\marginparsep}{7mm}
  \setlength{\marginparwidth}{20mm}
  \setlength{\headwidth}{\textwidth}
  \addtolength{\headwidth}{\marginparsep}
  \addtolength{\headwidth}{\marginparwidth}
  \addtolength{\marginparwidth}{15mm} % plus d'espace pour titres de documentation

  %% Titres des sections et sous-sections
  \setsecheadstyle{\normalfont\Large\sffamily\bfseries\raggedright}
  \setsubsecheadstyle{\normalfont\large\sffamily\bfseries\raggedright}
  \maxsecnumdepth{subsection}
  \setsecnumdepth{subsection}

  %% Listes. Paramétrage avec enumitem.
  \setlist[enumerate]{leftmargin=*,align=left}
  \setlist[enumerate,2]{label=\alph*),labelsep=*,leftmargin=1.5em}
  \setlist[enumerate,3]{label=\roman*),labelsep=*,leftmargin=1.5em,align=right}
  \setlist[itemize]{leftmargin=*,align=left}

  %% Paramétrage de babel
  \frenchbsetup{%
    StandardItemizeEnv=true,       % format standard des listes
    ThinSpaceInFrenchNumbers=true, % espace fine dans les nombres
    ItemLabeli=\mantriangleright,  % puce premier niveau
    ItemLabelii=\textendash,       % puce second niveau
    og=«, fg=»                     % caractères « et » sont les guillemets
  }
  \addto\captionsfrench{\def\figurename{{\scshape Fig.}}}
  \addto\captionsfrench{\def\tablename{{\scshape Tab.}}}
  \addto\captionsfrench{\def\listfigurename{Liste des figures}}

  %% Sections de code source
  \lstloadlanguages{[LaTeX]TeX}
  \lstset{language=[LaTeX]TeX,
    basicstyle=\ttfamily\NoAutoSpacing,
    keywordstyle=\mdseries,
    commentstyle=\color{comments}\slshape,
    emphstyle=\bfseries,
    escapeinside=`',
    extendedchars=true,
    showstringspaces=false,
    backgroundcolor=\color{LightYellow1},
    frame=leftline,
    framerule=2pt,
    framesep=5pt,
    xleftmargin=7.4pt
  }

  %%% =========================
  %%%  Nouveaux environnements
  %%% =========================

  %% Exemples
  \theoremstyle{definition}
  \newtheorem{exemple}{Exemple}[chapter]

  %% Exercices et réponses
  \Newassociation{sol}{solution}{solutions}
  \newcounter{exercice}[chapter]
  \renewcommand{\theexercice}{\thechapter.\arabic{exercice}}
  \newenvironment{exercice}[1][]{%
    \begin{list}{}{%
        \refstepcounter{exercice}
        \ifthenelse{\equal{#1}{nosol}}{%
          \renewcommand{\makelabel}{\bfseries\theexercice}}{%
          \hypertarget{ex:\theexercice}{}
          \Writetofile{solutions}{\protect\hypertarget{sol:\theexercice}{}}
          \renewcommand{\makelabel}{%
            \bfseries\protect\hyperlink{sol:\theexercice}{\theexercice}}}
        \settowidth{\labelwidth}{\bfseries\theexercice}
        \setlength{\leftmargin}{\labelwidth}
        \addtolength{\leftmargin}{\labelsep}
        \setlist[enumerate,1]{label=\alph*),labelsep=*,leftmargin=1.5em}
        \setlist[enumerate,2]{label=\roman*),labelsep=0.5em,align=right}}
      \item}
    {\end{list}}
  \renewenvironment{solution}[1]{%
    \begin{list}{}{%
        \renewcommand{\makelabel}{%
          \bfseries\protect\hyperlink{ex:#1}{#1}}
        \settowidth{\labelwidth}{\bfseries #1}
        \setlength{\leftmargin}{\labelwidth}
        \addtolength{\leftmargin}{\labelsep}
        \setlist[enumerate,1]{label=\alph*),labelsep=*,leftmargin=1.5em}
        \setlist[enumerate,2]{label=\roman*),labelsep=0.5em,align=right}}
      \item}
    {\end{list}}

  %% Démo de code LaTeX. Le code de 'texample' et 'eqxample' est
  %% repris de amsldoc.tex avec des petits ajustements.
  \newenvironment{demo}{%
    \begin{trivlist}\item}{%
    \end{trivlist}}
  \newenvironment{texample}[1][0.5\linewidth]{%
    \noindent\begin{minipage}{#1}%
      \def\producing{\end{minipage}\hfill\begin{minipage}{\dimexpr0.97\linewidth-#1}%
        \hbox\bgroup\kern-.2pt%
        \vbox\bgroup\parindent0pt\relax
        % The 3pt is to cancel the -\lineskip from \displ@y
        \abovedisplayskip3pt \abovedisplayshortskip\abovedisplayskip
        \belowdisplayskip0pt \belowdisplayshortskip\belowdisplayskip
        \noindent}
    }{%
      \par
      % Ensure that a lonely \[\] structure doesn't take up width less than
      % \hsize.
      \hrule height0pt width\hsize
      \egroup\kern-.2pt\egroup
    \end{minipage}%
    \par
  }
  \newenvironment{eqxample}{%
    \noindent\begin{minipage}{.5\columnwidth}%
      \def\producing{\end{minipage}\hfill\begin{minipage}{.45\columnwidth}%
        \hbox\bgroup\kern-.2pt\vrule width.2pt%
        \vbox\bgroup\parindent0pt\relax
        % The 3pt is to cancel the -\lineskip from \displ@y
        \abovedisplayskip3pt \abovedisplayshortskip\abovedisplayskip
        \belowdisplayskip0pt \belowdisplayshortskip\belowdisplayskip
        \noindent}
    }{%
      \par
      % Ensure that a lonely \[\] structure doesn't take up width less than
      % \hsize.
      \hrule height0pt width\hsize
      \egroup\vrule width.2pt\kern-.2pt\egroup
    \end{minipage}%
    \par
  }

  %% Un exemple du chapitre Trucs et astuces nécessite des
  %% environnements 'lstlisting' imbriqués, ce que ne digère pas
  %% LaTeX. La ruse consiste à définir un environnement équivalent qui
  %% porte simplement un autre nom.
  \lstnewenvironment{vglisting}{\lstset{deletetexcs={int,include}}}{}

  %% Exemples de notices bibliographiques
  \newenvironment{bibexample}[1][\linewidth]{%
    \begin{minipage}[t]{#1}%
      \begin{trivlist}}
      {\end{trivlist}\end{minipage}}

  %%% =======
  %%%  Index
  %%% =======
  \renewcommand{\preindexhook}{%
    Cet index contient des références aux commandes et environnements
    {\LaTeX}, ainsi qu'aux noms de paquetages et de classes. %
    Le premier numéro indique habituellement, mais pas toujours,
    la page où un concept est introduit, défini ou expliqué.%
    \vskip\onelineskip}
  \lstset{language=[AlLaTeX]TeX,
    morekeywords={align,align*,aligned,bmatrix,cases,equation*,%
      figure,gather,lstlisting,multline,quote,split,%
      table,tabular,tabularx},
    deletekeywords={document},   % répéter dans deletetexcs
    moretexcs={toprule,midrule,bottomrule,%
      includegraphics,reflectbox,resizebox,rotatebox,scalebox,%
      includepdf,frenchfigurename,frenchtablename,%
      newsubfloat,subcaption,%
      bm,dfrac,tfrac,iint,text,mathcal,mathbb,eqref,symbf,%
      citet,citep,citeauthor,citeyear,%
      setmainfont,setsansfont,setmonofont,setmathfont,%
      color,textcolor,definecolor,colorlet,hypersetup},
    deletetexcs={document,documentclass,usepackage,begin,end,LaTeX,TeX,%
      normalfont,bfseries,textbf,itshape,scshape,sffamily,ttfamily,texttt,%
      emph,small,Huge,raggedright,%
      hfill,def,a,b,c,d,em,i,j,l,r,t},
    index=[1][keywords],        % environnements
    indexstyle=[1]\ixenv,
    index=[2][texcs],           % commandes
    indexstyle=[2]\ixcmd}
  \newcommand{\ixenv}[1]{\index{#1 env@\Pe{#1}}%
    \index{environnement!#1@\Pe{#1}}}
  \newcommand{\ixcmd}[1]{\index{#1@\string\cs{#1}}}
  \makeindex

  %%% =====================
  %%%  Nouvelles commandes
  %%% =====================

  %% Noms de fonctions, code, environnement, etc.
  \newcommand{\code}[1]{\texttt{#1}}
  \newcommand{\fichier}[1]{\texttt{#1}}
  \newcommand{\class}[1]{\textsf{#1}\index{#1 class@\textsf{#1}}%
    \index{classe!#1}}
  \newcommand{\pkg}[1]{\textbf{#1}\index{#1 pkg@\textbf{#1}}%
    \index{paquetage!#1}}
  \newcommand{\Pe}[1]{\code{#1}}         % tiré de la doc de memoir
  \newcommand{\Ie}[1]{\Pe{#1}\ixenv{#1}} % idem
  \newcommand{\mat}[1]{\symbf{#1}}       % en mode mathématique

  %% Modification de commandes tirées de memoir.cls servant à afficher
  %% des noms de commandes.
  %% - \cmdprint est modifiée pour que le nom de la commande ne soit
  %%   pas en italique;
  %% - \cmd est modifiée pour utiliser @ comme séparateur dans \index
  %%   et pour utiliser \cs plutôt que \cmdprint pour afficher le nom de
  %%   la commande (afin d'obtenir le même format d'entrée d'index
  %%   qu'avec \ixcmd ci-dessus);
  %% - \pixbsbs et \pixabang servent respectivement à afficher et
  %%   indexer \\ et \! ;
  %% - \pixbar sert à afficher et indexer \| avec un hack pour
  %%   contourner un problème d'insertion de l'hyperlien vers le
  %%   numéro de page: l'entrée est triée sur le symbole [ plutôt que
  %%   sur |.
  \renewcommand{\cmdprint}[1]{\textup{\texttt{\string#1}}}
  \makeatletter
  \renewcommand{\cmd}[1]{\cmdprint{#1}%
    \index{\expandafter\@gobble\string#1@\string\cs{\expandafter\@gobble\string#1}}}
  \makeatother
  \newcommand*{\pixbsbs}{\cmdprint{\\}\index{"\ @\string\cs{}\bs}}
  \newcommand*{\pixabang}{\cmdprint{\!}\index{"!@\string\cs{}\texttt{"!}}}
  \newcommand*{\pixbar}{\cmdprint{\|}\index{[@\string\cs{}\texttt{\textbar}}}

  %% Boite additionnelle (basée sur awesomebox.sty) pour liens vers
  %% des vidéos
  \newcommand{\videobox}[1]{%
    \awesomebox{\aweboxrulewidth}{\faYoutube}{url}{#1}}

  %% Hyperlien avec symbole de lien externe juste après
  \newcommand{\link}[2]{\href{#1}{#2~\smaller\faExternalLink*}}

  %% Lien vers documentation dans la marge
  %% usage: \doc[documentation]{nom_fichier}{url}
  \newcommand{\doc}[3][documentation]{\link{#3}{#1}%
    \ifthenelse{\equal{#2}{}}{}{\marginpar%
      [\hfill\faBookmark~\fichier{#2}]%
      {\faBookmark~\fichier{#2}}}}

  %% Suppression de l'hyperlien
  \newcommand{\nolink}[1]{\begin{NoHyper}#1\end{NoHyper}}

  %% Lien vers Gitlab dans la page de notices
  \newcommand{\viewsource}[1]{%
    \href{#1}{\faGitlab\ Voir sur Gitlab}}

  %% Pour le tableau des commandes d'espacement en mode mathématique.
  %% Pris de la doc de amsmath.
  \newcommand{\lspx}{\mathord{\dashv\mkern-3mu}}
  \newcommand{\rspx}{\mathord{\mkern-2mu\vdash}}
  \newcommand{\spx}[1]{$\lspx #1\rspx$}

  %% Logo BIBTeX; tiré de https://bit.ly/1RQqUnG
  \newcommand{\BibTeX}{\rmfamily B\kern-.05em{\scshape i\kern-.025em b}\kern-.08em \TeX}

  %% Chapitre sur les bibliographies: des références bibliographiques
  %% sont insérées dans le texte avec \bibentry. Certaines commandes
  %% de francaisbst.tex sont alors utilisées, mais non encore
  %% définies. Répétées ici. De plus, il faut définir ici la commande
  %% \enquote plutôt que dans francais.bst. C'est pourquoi il y a une
  %% version modifiée de ce fichier dans ces sources.
  %% Voir https://bit.ly/1MORZmp
  \global\def\bbland{et}
  \global\def\bbledn{\'ed.}
  \global\def\bblfourtho{4{\ieme}}
  \global\def\bblth{{\ieme}}
  \global\def\bblvol{vol.}
  \def\bblno{\no{}}
  \def\bblpp{p.}
  \newcommand{\enquote}[1]{\guillemotleft#1\guillemotright}

  %%% =======
  %%%  Varia
  %%% =======

  %% Style de la bibliographie
  \bibliographystyle{francais}

  %% Longueurs pour la composition des pages couvertures avant et
  %% arrière.
  \newlength{\banderougewidth} \newlength{\banderougeheight}
  \newlength{\bandeorwidth}    \newlength{\bandeorheight}
  \newlength{\imageheight}
  \newlength{\logoheight}
  \newlength{\gapwidth}

%  \includeonly{}

\begin{document}

\frontmatter

%% Page couverture avant.
\pagestyle{empty}
\input{couverture-avant}
\null\cleardoublepage           % cf. section 2.2 textpos.pdf

%% Page de copyright
%%% Copyright (C) 2020 Vincent Goulet
%%%
%%% Ce fichier fait partie du projet
%%% «Rédaction avec LaTeX»
%%% https://gitlab.com/vigou3/formation-latex-ul
%%%
%%% Cette création est mise à disposition sous licence
%%% Attribution-Partage dans les mêmes conditions 4.0
%%% International de Creative Commons.
%%% https://creativecommons.org/licenses/by-sa/4.0/

\begingroup
\calccentering{\unitlength}
\begin{adjustwidth*}{\unitlength}{-\unitlength}
  \setlength{\parindent}{0pt}
  \setlength{\parskip}{\baselineskip}

  \raisebox{-2.5mm}{\includegraphics[height=7mm,keepaspectratio=true]{images/by-sa}} %
  {\theauthor}, {\year}

  {\textcopyright} {\year} par {\theauthor}. «\thetitle» est mis à
  disposition sous licence
  \href{https://creativecommons.org/licenses/by-sa/4.0/deed.fr}{%
    Attribution-Partage dans les mêmes conditions 4.0 International}
  de Creative Commons. En vertu de cette licence, vous êtes autorisé
  à:
  \begin{itemize}
  \item \textbf{partager} --- copier, distribuer et communiquer le
    matériel par tous moyens et sous tous formats;
  \item \textbf{adapter} --- remixer, transformer et créer à partir du
    matériel pour toute utilisation, y compris commerciale.
  \end{itemize}
  L'Offrant ne peut retirer les autorisations concédées par la licence
  tant que vous appliquez les termes de cette licence.

  Selon les conditions suivantes: \par
  \begin{tabularx}{\linewidth}{@{}lX@{}}
    \raisebox{-9mm}[0mm][13mm]{\includegraphics[height=11mm,keepaspectratio=true]{images/by}}
    & \textbf{Attribution} --- Vous devez créditer l'œuvre, intégrer
      un lien vers la licence et indiquer si des modifications ont été
      effectuées à l'œuvre. Vous devez indiquer ces informations par
      tous les moyens raisonnables, sans toutefois suggérer que
      l'Offrant vous soutient ou soutient la façon dont vous avez utilisé
      son œuvre. \\
    \raisebox{-9mm}{\includegraphics[height=11mm,keepaspectratio=true]{images/sa}}
    & \textbf{Partage dans les mêmes conditions} --- Dans le cas où vous
      effectuez un remix, que vous transformez, ou créez à partir du
      matériel composant l'œuvre originale, vous devez diffuser l'œuvre modifiée dans
      les mêmes conditions, c'est-à-dire avec la même licence avec laquelle
      l'œuvre originale a été diffusée.
  \end{tabularx}

  \textbf{Code source} \\
  \viewsource{\reposurl}

  \textbf{Couverture} \\
  Suricates (\emph{Suricata suricatta}) en Namibie. Parfois surnommé
  «sentinelle du désert», ce petit carnivore vit dans le sud-ouest de
  l'Afrique. Très prolifique, le suricate vit en grands groupes
  familiaux au sein d'une colonie. Crédit photo: {\textcopyright}
  Sara\&Joachim\&Mebe,
  \href{https://creativecommons.org/licenses/by-sa/2.0/deed.fr}{CC
    BY-SA 2.0} via
  \href{https://commons.wikimedia.org/w/index.php?curid=15305478}{%
    Wikimedia Commons}.

  Concept original du titre: Marie-Ève Guérard.
\end{adjustwidth*}
\endgroup

%%% Local Variables:
%%% mode: latex
%%% TeX-master: "formation-latex-ul"
%%% TeX-engine: xetex
%%% coding: utf-8
%%% End:

\clearpage

%% Corps du document
\pagestyle{companion}

%%% Copyright (C) 2020 Vincent Goulet
%%%
%%% Ce fichier fait partie du projet
%%% «Rédaction avec LaTeX»
%%% https://gitlab.com/vigou3/formation-latex-ul
%%%
%%% Cette création est mise à disposition sous licence
%%% Attribution-Partage dans les mêmes conditions 4.0
%%% International de Creative Commons.
%%% https://creativecommons.org/licenses/by-sa/4.0/

\chapter{Introduction}
\label{chap:introduction}

Le présent ouvrage tire son origine d'une formation sur la rédaction
de thèses et de mémoires avec {\LaTeX} développée pour la Bibliothèque
de l'Université Laval. La formation aborde les concepts de base pour
un nouvel utilisateur: processus d'édition, compilation,
visualisation; séparation du contenu et de l'apparence du texte; mise
en forme du texte; séparation du document en parties; rudiments du
mode mathématique. Transformée en prose, la série de diapositives qui
appuie la présentation correspond grosso modo aux quatre premiers
chapitres de l'ouvrage.

Les six autres chapitres visent à rendre l'utilisateur de {\LaTeX}
débutant ou intermédiaire autonome dans la rédaction de documents
relativement complexes comportant des tableaux, des figures, des
équations mathématiques élaborées, une bibliographie, etc. Nous avons
aussi émaillé le texte de conseils et d'astuces glanés au fil de nos
vingt années d'utilisation du système de mise en page.

Les nombreuses références à la classe de documents \class{ulthese}
s'adressent au premier public de l'ouvrage: les étudiantes et
étudiants de l'Université Laval occupés à la rédaction de leur thèse
ou de leur mémoire. Ils devront utiliser cette classe pour composer un
document conforme aux règles générales de présentation matérielle de
la Faculté des études supérieures et postdoctorales. Les autres
lecteurs pourront sans mal escamoter ces passages.

Chaque chapitre comporte quelques exercices. Les solutions se trouvent
en annexe.

\section*{Fonctionnalités interactives}

En consultation électronique, ce document se trouve enrichi de
plusieurs fonctionnalités interactives.
\begin{itemize}
\item Intraliens entre le numéro d'un exercice et sa solution, et vice
  versa. Ces intraliens sont aussi marqués par la couleur
  \textcolor{link}{\rule{1.5em}{1.2ex}}.
\item Intraliens entre les citations dans le texte et leur entrée dans
  la bibliographie. Ces intraliens sont marqués par la couleur
  \textcolor{citation}{\rule{1.5em}{1.2ex}}.
\item Hyperliens vers des ressources externes marqués par le symbole
  {\faExternalLink*} et la couleur
  \textcolor{url}{\rule{1.5em}{1.2ex}}.
\item Table des matières, liste des tableaux et liste des figures
  permettant d'accéder rapidement à des ressources du document.
\item Index comprenant les références aux commandes et environnements
  {\LaTeX}, ainsi qu'aux noms de paquetages et de classes.
\end{itemize}

\section*{Blocs signalétiques}

Le document est parsemé de divers types de blocs signalétiques
inspirés de
\link{https://asciidoctor.org/docs/user-manual/\#admonition}{AsciiDoc}
qui visent à attirer votre attention sur une notion.

\tipbox{Astuce! Ces blocs contiennent un truc, une astuce, ou tout
  autre type d'information utile.}
\vspace{-\baselineskip}

\importantbox{Important! Ces blocs contiennent les remarques les plus
  importantes. Veillez à en tenir compte.}
\vspace{-\baselineskip}

\notebox{Ces blocs contiennent des remarques additionnelles sur
  la matière ou des informations amusantes, mais non essentielles.}
\vspace{-\baselineskip}

\videobox{Ces blocs contiennent des renvois vers des vidéos dans la %
  \link{https://www.youtube.com/channel/UCIQia52CHKSCQG3_mwLSNCA}{chaine
    YouTube} liée ce document.}

\section*{Liens vers la documentation}

Parmi les hyperliens vers des ressources externes, le texte compte
plusieurs renvois vers la documentation d'un paquetage ou d'une
classe, par exemple vers la %
\doc{memoir}{http://texdoc.net/pkg/memoir/} %
de la classe \class{memoir}. L'hyperlien mène vers la documentation
dans le site %
\link{http://texdoc.net}{TeXdoc Online}. On trouve également dans la
marge le nom du fichier correspondant (sans l'extension \code{.pdf})
sur un système doté de {\TeX}~Live.

La plupart des logiciels intégrés de rédaction {\LaTeX} offrent une
interface pour accéder à cette documentation.
\begin{itemize}
\item TeXShop: menu \code{Aide|Afficher l'aide pour le
    package} (\optkey\,\cmdkey\, I).
\item Texmaker: menu \code{Aide|TeXDoc [selection]}.
\item GNU~Emacs: commande \code{TeX-doc} (\code{C-c ?}) du mode
  spécialisé AUC{\TeX}.
\end{itemize}
Le lecteur devrait consulter la rubrique d'aide de son éditeur pour
savoir s'il offre une interface au système de gestion de la
documentation Texdoc de {\TeX}~Live.

\section*{Références additionnelles}

L'ouvrage n'a aucune prétention d'exhaustivité. La consultation de
documentation additionnelle pourra s'avérer nécessaire pour réaliser
des mises en page plus élaborées. À cet égard, nous recommandons
chaudement le livre de \citet{Kopka:latex:4e} --- il a servi
d'inspiration pour ce document à maints endroits. La très complète
documentation (plus de 600~pages!) de la classe \class{memoir}
\citep{memoir} constitue une autre référence de choix. Nous
recommandons également:
\begin{itemize}
\item \link{https://fr.wikibooks.org/wiki/LaTeX}{\emph{LaTeX} dans
    Wikilivre} pour de la documentation en ligne, en français et
  libre;
\item le très actif forum de discussion
  \link{https://tex.stackexchange.com}{{\TeX}--{\LaTeX} Stack Exchange}
  (avant de penser y poser une question, vérifier que la réponse ne se trouve
  pas déjà dans le forum\dots\ ce qui risque fort d'être le cas);
\item la très complète
  \link{http://www.tex.ac.uk/cgi-bin/texfaq2html}{%
    \emph{foire aux questions}} (en anglais) du groupe des
  utilisateurs de {\LaTeX} du Royaume-Uni.
\end{itemize}

\section*{Installation d'une distribution}

L'utilisation de {\LaTeX} requiert évidemment une distribution du
système. Nous recommandons la distribution {\TeX}~Live administrée par
le {\TeX} Users Group.

\videobox{Visionnez les vidéos qui expliquent comment installer la
  distribution {\TeX}~Live \link{https://youtu.be/uJFbhQkDbU8}{sur
    macOS} et \link{https://youtu.be/wO2FlNmye14}{sur Windows}.}

\section*{Fichiers d'accompagnement}

Ce document devrait être accompagné des fichiers nécessaires pour
compléter certains exercices figurant à la fin des chapitres, ainsi
que d'un gabarit \fichier{exercice\_gabarit.tex} pour composer les
solutions des autres exercices. Si ce n'est pas le cas, récupérer les
fichiers dans le site \link{\ctanurl}{\emph{Comprehensive TeX Archive
    Network} (CTAN)}.

\section*{Document libre}

Tout comme {\TeX}, {\LaTeX} et l'ensemble des outils présentés dans ce
document, le projet «Rédaction avec {\LaTeX}» s'inscrit dans le
mouvement de
l'\link{https://www.gnu.org/philosophy/free-sw.html}{informatique
  libre}. Vous pouvez accéder à l'ensemble du code source en suivant
le lien dans la page de copyright. Vous trouverez dans le fichier
\code{README.md} toutes les informations utiles pour composer le
document.

%%% Local Variables:
%%% mode: latex
%%% TeX-master: "formation-latex-ul"
%%% TeX-engine: xetex
%%% encoding: utf-8
%%% End:


\cleartorecto
\tableofcontents
\cleartorecto
\listoftables
\cleartorecto
\listoffigures

\mainmatter

\include{presentation}
\include{bases}
\include{organisation}
\include{apparence}
\include{boites}
\include{tableaux+figures}
\include{mathematiques}
\include{bibliographie}
\include{commandes}
\include{trucs+astuces}

\appendix
\include{ulthese}               % éléments spécifiques à ulthese
%%% Copyright (C) 2020 Vincent Goulet
%%%
%%% Ce fichier fait partie du projet
%%% «Rédaction avec LaTeX»
%%% https://gitlab.com/vigou3/formation-latex-ul
%%%
%%% Cette création est mise à disposition sous licence
%%% Attribution-Partage dans les mêmes conditions 4.0
%%% International de Creative Commons.
%%% https://creativecommons.org/licenses/by-sa/4.0/

\chapter{Solutions des exercices}
\label{chap:solutions}

\input{solutions-bases}
% \input{solutions-organisation}
% \input{solutions-apparence}
\input{solutions-boites}
\input{solutions-tableaux+figures}
\input{solutions-mathematiques}
\input{solutions-bibliographie}
\input{solutions-commandes}
\input{solutions-trucs+astuces}

%%% Local Variables:
%%% mode: latex
%%% TeX-master: "formation-latex-ul"
%%% TeX-engine: xetex
%%% coding: utf-8
%%% End:


\bibliography{formation-latex-ul}

\cleardoublepage
\printindex

\cleartoverso

\input{colophon}

\cleartoverso

%% Page couverture arrière.
\pagestyle{empty}
\input{couverture-arriere}

\end{document}

%%% Local Variables:
%%% mode: latex
%%% TeX-engine: xetex
%%% TeX-master: t
%%% coding: utf-8
%%% End:
