%%% Copyright (C) 2020 Vincent Goulet
%%%
%%% Ce fichier fait partie du projet
%%% «Rédaction avec LaTeX»
%%% https://gitlab.com/vigou3/formation-latex-ul
%%%
%%% Cette création est mise à disposition sous licence
%%% Attribution-Partage dans les mêmes conditions 4.0
%%% International de Creative Commons.
%%% https://creativecommons.org/licenses/by-sa/4.0/

\chapter{Introduction}
\label{chap:introduction}

Le présent ouvrage tire son origine d'une formation sur la rédaction
de thèses et de mémoires avec {\LaTeX} développée pour la Bibliothèque
de l'Université Laval. La formation aborde les concepts de base pour
un nouvel utilisateur: processus d'édition, compilation,
visualisation; séparation du contenu et de l'apparence du texte; mise
en forme du texte; séparation du document en parties; rudiments du
mode mathématique. Transformée en prose, la série de diapositives qui
appuie la présentation correspond grosso modo aux quatre premiers
chapitres de l'ouvrage.

Les six autres chapitres visent à rendre l'utilisateur de {\LaTeX}
débutant ou intermédiaire autonome dans la rédaction de documents
relativement complexes comportant des tableaux, des figures, des
équations mathématiques élaborées, une bibliographie, etc. Nous avons
aussi émaillé le texte de conseils et d'astuces glanés au fil de nos
vingt années d'utilisation du système de mise en page.

Les nombreuses références à la classe de documents \class{ulthese}
s'adressent au premier public de l'ouvrage: les étudiantes et
étudiants de l'Université Laval occupés à la rédaction de leur thèse
ou de leur mémoire. Ils devront utiliser cette classe pour composer un
document conforme aux règles générales de présentation matérielle de
la Faculté des études supérieures et postdoctorales. Les autres
lecteurs pourront sans mal escamoter ces passages.

Chaque chapitre comporte quelques exercices. Les solutions se trouvent
en annexe.

\section*{Fonctionnalités interactives}

En consultation électronique, ce document se trouve enrichi de
plusieurs fonctionnalités interactives.
\begin{itemize}
\item Intraliens entre le numéro d'un exercice et sa solution, et vice
  versa. Ces intraliens sont aussi marqués par la couleur
  \textcolor{link}{\rule{1.5em}{1.2ex}}.
\item Intraliens entre les citations dans le texte et leur entrée dans
  la bibliographie. Ces intraliens sont marqués par la couleur
  \textcolor{citation}{\rule{1.5em}{1.2ex}}.
\item Hyperliens vers des ressources externes marqués par le symbole
  {\faExternalLink*} et la couleur
  \textcolor{url}{\rule{1.5em}{1.2ex}}.
\item Table des matières, liste des tableaux et liste des figures
  permettant d'accéder rapidement à des ressources du document.
\item Index comprenant les références aux commandes et environnements
  {\LaTeX}, ainsi qu'aux noms de paquetages et de classes.
\end{itemize}

\section*{Blocs signalétiques}

Le document est parsemé de divers types de blocs signalétiques
inspirés de
\link{https://asciidoctor.org/docs/user-manual/\#admonition}{AsciiDoc}
qui visent à attirer votre attention sur une notion.

\tipbox{Astuce! Ces blocs contiennent un truc, une astuce, ou tout
  autre type d'information utile.}
\vspace{-\baselineskip}

\importantbox{Important! Ces blocs contiennent les remarques les plus
  importantes. Veillez à en tenir compte.}
\vspace{-\baselineskip}

\notebox{Ces blocs contiennent des remarques additionnelles sur
  la matière ou des informations amusantes, mais non essentielles.}
\vspace{-\baselineskip}

\videobox{Ces blocs contiennent des renvois vers des vidéos dans la %
  \link{https://www.youtube.com/channel/UCIQia52CHKSCQG3_mwLSNCA}{chaine
    YouTube} liée ce document.}

\section*{Liens vers la documentation}

Parmi les hyperliens vers des ressources externes, le texte compte
plusieurs renvois vers la documentation d'un paquetage ou d'une
classe, par exemple vers la %
\doc{memoir}{http://texdoc.net/pkg/memoir/} %
de la classe \class{memoir}. L'hyperlien mène vers la documentation
dans le site %
\link{http://texdoc.net}{TeXdoc Online}. On trouve également dans la
marge le nom du fichier correspondant (sans l'extension \code{.pdf})
sur un système doté de {\TeX}~Live.

La plupart des logiciels intégrés de rédaction {\LaTeX} offrent une
interface pour accéder à cette documentation.
\begin{itemize}
\item TeXShop: menu \code{Aide|Afficher l'aide pour le
    package} (\optkey\,\cmdkey\, I).
\item Texmaker: menu \code{Aide|TeXDoc [selection]}.
\item GNU~Emacs: commande \code{TeX-doc} (\code{C-c ?}) du mode
  spécialisé AUC{\TeX}.
\end{itemize}
Le lecteur devrait consulter la rubrique d'aide de son éditeur pour
savoir s'il offre une interface au système de gestion de la
documentation Texdoc de {\TeX}~Live.

\section*{Références additionnelles}

L'ouvrage n'a aucune prétention d'exhaustivité. La consultation de
documentation additionnelle pourra s'avérer nécessaire pour réaliser
des mises en page plus élaborées. À cet égard, nous recommandons
chaudement le livre de \citet{Kopka:latex:4e} --- il a servi
d'inspiration pour ce document à maints endroits. La très complète
documentation (plus de 600~pages!) de la classe \class{memoir}
\citep{memoir} constitue une autre référence de choix. Nous
recommandons également:
\begin{itemize}
\item \link{https://fr.wikibooks.org/wiki/LaTeX}{\emph{LaTeX} dans
    Wikilivre} pour de la documentation en ligne, en français et
  libre;
\item le très actif forum de discussion
  \link{https://tex.stackexchange.com}{{\TeX}--{\LaTeX} Stack Exchange}
  (avant de penser y poser une question, vérifier que la réponse ne se trouve
  pas déjà dans le forum\dots\ ce qui risque fort d'être le cas);
\item la très complète
  \link{http://www.tex.ac.uk/cgi-bin/texfaq2html}{%
    \emph{foire aux questions}} (en anglais) du groupe des
  utilisateurs de {\LaTeX} du Royaume-Uni.
\end{itemize}

\section*{Installation d'une distribution}

L'utilisation de {\LaTeX} requiert évidemment une distribution du
système. Nous recommandons la distribution {\TeX}~Live administrée par
le {\TeX} Users Group.

\videobox{Visionnez les vidéos qui expliquent comment installer la
  distribution {\TeX}~Live \link{https://youtu.be/uJFbhQkDbU8}{sur
    macOS} et \link{https://youtu.be/wO2FlNmye14}{sur Windows}.}

\section*{Fichiers d'accompagnement}

Ce document devrait être accompagné des fichiers nécessaires pour
compléter certains exercices figurant à la fin des chapitres, ainsi
que d'un gabarit \fichier{exercice\_gabarit.tex} pour composer les
solutions des autres exercices. Si ce n'est pas le cas, récupérer les
fichiers dans le site \link{\ctanurl}{\emph{Comprehensive TeX Archive
    Network} (CTAN)}.

\section*{Document libre}

Tout comme {\TeX}, {\LaTeX} et l'ensemble des outils présentés dans ce
document, le projet «Rédaction avec {\LaTeX}» s'inscrit dans le
mouvement de
l'\link{https://www.gnu.org/philosophy/free-sw.html}{informatique
  libre}. Vous pouvez accéder à l'ensemble du code source en suivant
le lien dans la page de copyright. Vous trouverez dans le fichier
\code{README.md} toutes les informations utiles pour composer le
document.

%%% Local Variables:
%%% mode: latex
%%% TeX-master: "formation-latex-ul"
%%% TeX-engine: xetex
%%% encoding: utf-8
%%% End:
