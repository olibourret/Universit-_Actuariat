%%% Copyright (C) 2020 Vincent Goulet
%%%
%%% Ce fichier fait partie du projet
%%% «Rédaction avec LaTeX»
%%% https://gitlab.com/vigou3/formation-latex-ul
%%%
%%% Cette création est mise à disposition sous licence
%%% Attribution-Partage dans les mêmes conditions 4.0
%%% International de Creative Commons.
%%% https://creativecommons.org/licenses/by-sa/4.0/

\begin{frame}[t,plain,fragile=singleslide]
  \tiny
  \vspace*{10mm}

  \begin{adjustwidth}{10mm}{10mm}
    \raisebox{-1.4mm}{\includegraphics[height=4mm,keepaspectratio=true]{images/by-sa}} %
    Vincent Goulet, {\year}

    {\textcopyright} {\year} par Vincent Goulet. «Rédaction avec
    {\LaTeX} --- Premiers pas» est mis à disposition sous licence
    \href{https://creativecommons.org/licenses/by-sa/4.0/deed.fr}{%
      Attribution-Partage dans les mêmes conditions 4.0 International}
    de Creative Commons. En vertu de cette licence, vous êtes autorisé
    à:
    \begin{itemize}
    \item \textbf{partager} --- copier, distribuer et communiquer le
      matériel par tous moyens et sous tous formats;
    \item \textbf{adapter} --- remixer, transformer et créer à partir du
      matériel pour toute utilisation, y compris commerciale.
    \end{itemize}
    L'Offrant ne peut retirer les autorisations concédées par la licence
    tant que vous appliquez les termes de cette licence.

    Selon les conditions suivantes: \par
    \begin{tabularx}{\linewidth}{@{}lX@{}}
      \raisebox{-5.5mm}[0mm][10mm]{\includegraphics[height=7mm,keepaspectratio=true]{images/by}}
      & \textbf{Attribution} --- Vous devez créditer l'œuvre, intégrer
        un lien vers la licence et indiquer si des modifications ont été
        effectuées à l'œuvre. Vous devez indiquer ces informations par
        tous les moyens raisonnables, sans toutefois suggérer que
        l'Offrant vous soutient ou soutient la façon dont vous avez utilisé
        son œuvre. \\
      \raisebox{-5.5mm}{\includegraphics[height=7mm,keepaspectratio=true]{images/sa}}
      & \textbf{Partage dans les mêmes conditions} --- Dans le cas où vous
        effectuez un remix, que vous transformez, ou créez à partir du
        matériel composant l'œuvre originale, vous devez diffuser l'œuvre modifiée dans
        les mêmes conditions, c'est-à-dire avec la même licence avec laquelle
        l'œuvre originale a été diffusée.
    \end{tabularx}

    \textbf{Code source} \\
    \viewsource{\reposurl}

    \textbf{Couverture} \\
    Suricates (\emph{Suricata suricatta}) en Namibie. Parfois surnommé
    «sentinelle du désert», ce petit carnivore vit dans le sud-ouest
    de l'Afrique. Très prolifique, le suricate vit en grands groupes
    familiaux au sein d'une colonie. Crédit photo: {\textcopyright}
    Sara\&Joachim\&Mebe,
    \href{https://creativecommons.org/licenses/by-sa/2.0/deed.fr}{CC
      BY-SA 2.0} via
    \href{https://commons.wikimedia.org/w/index.php?curid=15305478}{%
      Wikimedia Commons}.

    Concept original du titre: Marie-Ève Guérard.
  \end{adjustwidth}
\end{frame}

%%% Local Variables:
%%% TeX-master: "formation-latex-ul-diapos"
%%% TeX-engine: xetex
%%% coding: utf-8
%%% End:
